% !Mode:: "TeX:UTF-8"
% !TEX TS-program = XeLaTeX
% !TEX encoding = UTF-8 Unicode

%%%%%%%%%%%%%%%%%%%%%%%%%%%%%%%%%%%%%%%%%%%%%%%%%%%%%%%%%%%%%%%%%%%%%%
%
%	东北大学博士论文 XeLaTeX 模版 —— 封面文件 cover.tex
%	版本:0.71
%	最后更新:2010.12.22
%	修改者:Yuri (E-mail: yuri_1985@163.com)
%   二次修订:bainhome (maliang7653@sina.com)
%	编译环境: Ubuntu 10.04 + TeXLive 2010 + TeXworks
%   编译环境2:Windows 7 SP1 + TeXLive2016 + TeXStudio
%
%%%%%%%%%%%%%%%%%%%%%%%%%%%%%%%%%%%%%%%%%%%%%%%%%%%%%%%%%%%%%%%%%%%%%%

\cdegree{学\;\;\;\;位\;\;\;\;论\;\;\;\;文}
% ----------- * 填 * 写 * 个 * 人 * 信 * 息 * 构 * 造 * 论 * 文 * 封 * 面 * ---------
\ctitle{基于分数阶Fourier变换的玉女十九剑噪声光谱及轨迹预测模型}
\etitle{The \XeLaTeX{} Template of Doctor Degree Thesis  of NEU}

% 根据需要添加字符间距
\csubject{华山剑招理论及改良}              % 学科的专业中文名称,例如东北大学机械工程与自动化学院
\cauthor{令狐冲}                          % 中文作者姓名
\eauthor{Linghu Chong}                   % 英文作者姓名
%\cauthorno{31415926}
\csupervisor{岳不群\;掌门人}              % 导师中文姓名和职称,例如张三\;教授,中间的符号是一个小的间距
\esupervisor{Person in charge\;YUE Bu-qun}% 导师英文姓名和职称,例如张三\;教授,中间的符号是一个小的间距
\cNEUpartment{五岳联盟华山剑术学院}        % 东大的话,直接填东北大学加自己的学院名
\cNEU{剑术研究学院·黑木崖}                 %\cNEU{东~~~北~~~大~~~学}
\eNEU{Northeastern University}           % 学校英文名
\cMajor{实~~战}                          %\cMajor{工~~学}
\eMajor{Mechanical Design and Automation} % 专业英文名称
\cDegree{剑术大师}                        % 博士
\topNEU{\kai\wuhao{东北大学博士学位论文}} % 页眉学校标题

%% ----------- * 定 * 义 * 字 * 体 * 名 * 称 * --------------
%\fontMin{MinionPro-Regular}
%\fontGara{AGaramondPro-Regular}
%\fontT{Times New Roman}
%\fontArial{Arial}
%\fontAsong{Adobe Song Std}
%\fontAHei{Adobe Heiti Std}
%\fontAFs{Adobe Fangsong Std}
%\fontAKai{Adobe Kaiti Std}

%\funnynameA{方证(少林学院)}
%\funnynameB{丁不四,王重阳,丁典,阳顶天,冲虚}
\etitle{Research and Simulated Implementation of QoS Handoff Mechanisms Based on GA with ABC Supported}
% 这里默认使用最后编译的时间,也可自行给定日期,注意汉字和数字之间的空格。
\cdate{\the\year~年~\the\month~月~\the\day~日}
\tmdate{\the\year~年~\the\month~月}
\edate{\ifcase \month \or January\or February\or March\or April\or May%
       \or June\or July \or August\or September\or October\or November
       \or December\fi\unskip~ \the\year}

\cabstract{
2012年,艾均(东大信息学院)在清华薛瑞尼工作基础上,稍作调整并提交一份\hologo{pdfLaTeX} 编译方式的论文模板,原模板论文格式兼顾清华大学的学士、硕士、博士和博士后四种格式学位论文,对其他学校,尤其是仅仅针对博士论文撰写而言,这份模板的代码中有过多冗余部分,后期维护有一定挑战性,同时,\hologo{pdfLaTeX} 编译方式不易使用系统字体、过时的CJK语言宏包等等,也让该模板有点儿不太能跟上\TeX 编译的主流形式,虽然2016年3月薛瑞尼在GitHub对模板再次更新,兼顾\hologo{pdfLaTeX}和\hologo{XeLaTeX}两种编译方式,但版本变化确实较大,加之个人水平和时间有限,根据它定义东大模板感觉困难更大。因此,我参考北京大学、清华大学、哈尔滨工业大学,以及大连理工大学论文~\hologo{LaTeX} 模板,按东北大学博士学位论文格式规范,制订了属于东大的~\hologo{XeLaTeX}~论文模板,多数页面、字体设置的结构源自哈工大和大工的模板,对兄弟院校的\TeX 高手表示谢意和敬意。不过参考文献则在TeXLive2016下使用Bib\LaTeX ,其著录规则参照最新的GB/T7714-2015文献著录规则。经一定测试,大体满足东北大学论文规范要求,可能还存在一些问题,欢迎使用本模版,反馈遇到的问题,以便改进。希望这一模板能为在东大推广~\hologo{LaTeX}~,发挥作用。因为一般掌握简单的\hologo{TeX} 语法学习,就能利用本模板,在动辄一百页以上、文献引用量接近两百条的博士论文写作过程中,节省大量时间精力,并轻松地排出接近格式标准而优美的论文。

本模板中,采用的编译环境:\menu[,]{Windows7.0-SP1,TeXLive2016,TeXStudio},其中后两个都是跨平台的自由软件,相信在高版本Windows中也不存在太大问题。前端采用TeXStudio而不是TeXLive2016自带的TeXWorks,因为TeXWorks似乎没有多标签的设置,这对分章节撰写的博士论文不是特别方便。在写模板的过程中也用过CTeX套装,只是CTeX已经太长时间不更新了,一些比较有趣的宏包无法使用,例如Bib\LaTeX 、代码高亮宏包minted等等,都要手动安装,虽然这也并不造成本质性的障碍,但考虑到使用便利性,还是换成了2016年发行的TeXLive2016。
}

\ckeywords{博士学位论文;排版格式;Bib\hologo{LaTeX};\hologo{XeLaTeX}模版}

\eabstract{
This is a \LaTeX{} template of doctor degree thesis of Dalian University of Technology,
which is built according to the required format.

内容应与“中文摘要”对应。使用第三人称,最好采用现在时态编写。
}

\ekeywords{Write Criterion; Typeset Format; Doctor's Degree Paper; \XeLaTeX{} Template}

\makecover
