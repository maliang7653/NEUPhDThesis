% !Mode:: "TeX:UTF-8"
% !TEX TS-program = XeLaTeX
% !TEX encoding = UTF-8 Unicode

%%%%%%%%%%%%%%%%%%%%%%%%%%%%%%%%%%%%%%%%%%%%%%%%%%%%%%%%%%%%%%%%%%%%
%
% 东北大学博士论文 XeLaTeX 模版 —— 宏包配置文件 packages.tex
%	版本:0.21
%	最后更新:2017.03.16
%   二次修订:bainhome (maliang7653@sina.com)
%   编译环境:Windows 7 SP1 + TeXLive2016 + TeXStudio
%%%%%%%%%%%%%%%%%%%%%%%%%%%%%%%%%%%%%%%%%%%%%%%%%%%%%%%%%%%%%%%%%%%%%

\usepackage{expl3,ifthen,xstring}

% 页面设置
\usepackage[body={16.0cm, 24.7cm}]{geometry}
\usepackage{indentfirst}                         % 首行缩进宏包
\usepackage[sf]{titlesec}                        % 控制标题的宏包
\usepackage{titletoc}                            % 控制目录的宏包
\usepackage{titleref}                            % 标题引用
\usepackage{fancyhdr}                            % 自定义页眉页脚
\usepackage{fancyref}                            % 引用链接属性
\usepackage[perpage,symbol]{footmisc}            % 脚注控制
\usepackage{layouts}                             % 打印当前页面格式的宏包
\usepackage{paralist}                            % 一种换行不缩进的列表格式,asparaenum,inparaenum 等
\usepackage[shortlabels]{enumitem}               % 列表格式
\usepackage{fancyvrb}                            % 原样输出
\usepackage[amsmath,thmmarks,hyperref]{ntheorem} % 定理类环境宏包
\usepackage{type1cm}                             % 控制字体的大小

% 表格处理
\usepackage{booktabs}   % 三线表
\usepackage{multirow}   % 表格多行处理
\usepackage{diagbox}    % 斜线表头
\usepackage{tabularx}   % 表格折行
\usepackage{siunitx}    % 国际单位,小数点对齐

% ---------- * 数 * 学 * 相 * 关 * -------------
\usepackage{amsmath}
\usepackage{amssymb}
\usepackage{amsfonts}
\usepackage{mathrsfs}
\usepackage{amsthm}
\usepackage{amscd}

% ---------- * 特 * 殊 * 符 * 号 * 相 * 关 * -------------
\usepackage{rotating}
\usepackage{upgreek}
\usepackage{pifont}
\usepackage{bbding}
\usepackage{bm}


% ---------- * 图 * 形 * 相 * 关 * -------------
\usepackage{graphicx}         % 请在引用图片时务必给出后缀名
\usepackage[x11names]{xcolor} % 支持彩色
\usepackage[below]{placeins}  % 浮动图形控制宏包
\usepackage{rotating}	      % 图形和表格的控制
\usepackage{picinpar}
\usepackage{setspace}         % 定制表格和图形的多行标题行距
\usepackage{subfigure}           % 插入子图形
\usepackage[subfigure]{ccaption} % 插图表格的双语标题
\usepackage{overpic}

\usepackage{listings}         % 源代码展示
\lstset{%
  language=TeX,
  defaultdialect=empty,
  basicstyle=\ttfamily\small,
  backgroundcolor=\color{LightSteelBlue1},
  keywordstyle=\color{blue},
  showspaces=false,
  showstringspaces=false,
  showtabs=false,
  tabsize=2,breakatwhitespace=false,
  columns=flexible}
%\usepackage{verbatim}
% 其他
\usepackage{calc}   % 在 tex 文件中具有一些计算功能,主要用在页面控制。
%\usepackage[xetex,
%bookmarksnumbered=true,
%bookmarksopen=true,
%colorlinks=true,
%% pdfborder={0 0 1},
%citecolor=blue,
%linkcolor=blue,
%anchorcolor=green,
%urlcolor=magenta,
%breaklinks=true,
%CJKbookmarks=true,
%]{hyperref}

\usepackage[CJKbookmarks,
colorlinks=true,
bookmarksnumbered=true,%
pdfstartview=FitH,
linkcolor=blue,
anchorcolor=violet,
citecolor=magenta]{hyperref}  %书签功能,选项去掉链接红色方框

%\usepackage[numbers,sort&compress,square,super]{natbib} %参考文献
%\usepackage{hypernat}
%\usepackage{bibentry}

\usepackage[backend=biber,style=gb7714_2015,]{biblatex}
\setlength{\bibitemsep}{2pt}
%\defbibheading{bibliography}[\bibname]{%
%	\phantomsection%解决链接指引出错的问题,相当于加入了一个引导点
%\addcontentsline{toc}{subsubsection}{#1}
%\centering\subsubsection*{#1}}%
\defbibheading{bibliography}[\bibname]{%
	\chapter*{#1}%
	\markboth{#1}{#1}}
% ------------- * BibLaTeX * 文 * 献 * 设 * 置 * ---------------
%\addbibresource[location=local]{PHDReferences.bib}
\addbibresource[location=local]{Ref.bib}
%\usepackage{ulem}
%\newcommand\yellowback{\bgroup\markoverwith
%	{\textcolor{yellow}{\rule[-0.5ex]{2pt}{2.5ex}}}\ULon}
%\newcommand\reduline{\bgroup\markoverwith
%	{\textcolor{red}{\rule[-0.5ex]{2pt}{0.4pt}}}\ULon}

% ---------- * 文 * 件 * 夹 * 与 * 路 * 径 * logo * 显 * 示 * ------------
\usepackage{menukeys}
\usepackage{hologo}
\renewmenumacro{\directory}{pathswithfolder}










