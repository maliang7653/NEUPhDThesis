% !Mode:: "TeX:UTF-8"
% !TEX TS-program = XeLaTeX
% !TEX encoding = UTF-8 Unicode

%%%%%%%%%%%%%%%%%%%%%%%%%%%%%%%%%%%%%%%%%%%%%%%%%%%%%%%%%%%%%%%%%%%%
%
%   东北大学博士论文 XeLaTeX 模版 —— 宏包配置文件 packages.tex
%	版本:0.2
%	最后更新:2017.02.16
%   二次修订:bainhome (E-main: maliang7653@sina.com)
%   编译环境:Windows 7 SP1 + CTeXLive 2016 + TeXStudio2.12.2
%%%%%%%%%%%%%%%%%%%%%%%%%%%%%%%%%%%%%%%%%%%%%%%%%%%%%%%%%%%%%%%%%%%%%

\usepackage{expl3,ifthen,xstring}

% 页面设置
\usepackage[body={16.0cm, 24.7cm}]{geometry}
\usepackage{indentfirst}                         % 首行缩进宏包
\usepackage[sf]{titlesec}                        % 控制标题的宏包
\usepackage{titletoc}                            % 控制目录的宏包
\usepackage{titleref}                            % 标题引用
\usepackage{fancyhdr}                            % 自定义页眉页脚
\usepackage{fancyref}                            % 引用链接属性
\usepackage[perpage,symbol]{footmisc}            % 脚注控制
\usepackage{layouts}                             % 打印当前页面格式的宏包
\usepackage{paralist}                            % 一种换行不缩进的列表格式,asparaenum,inparaenum 等
\usepackage[shortlabels]{enumitem}               % 列表格式
\usepackage{fancyvrb}                            % 原样输出
\usepackage[amsmath,thmmarks,hyperref]{ntheorem} % 定理类环境宏包
\usepackage{type1cm}                             % 控制字体的大小
\usepackage{lscape}                              % 控制单页横向

% 表格处理
\usepackage{booktabs}   % 三线表
\usepackage{multirow}   % 表格多行处理
\usepackage{diagbox}    % 斜线表头
\usepackage{tabularx}   % 表格折行
\usepackage{siunitx}    % 国际单位,小数点对齐
\usepackage{threeparttable}

% ---------- * 数 * 学 * 相 * 关 * -------------
\usepackage{amsfonts}
\usepackage{mathrsfs}
\usepackage{amsthm}
\usepackage{amscd}
% ---------- * 新 * 数 * 学 * 字 * 体 * 测 * 试 * -------------
\usepackage[lite,subscriptcorrection,slantedGreek,nofontinfo]{mtpro2}
% ---------------- * 化 * 学 * 相 * 关 * 宏 * 包 * -------------
\usepackage[version=4]{mhchem}

% ---------- * 特 * 殊 * 符 * 号 * 相 * 关 * -------------
\usepackage{rotating}
\usepackage{upgreek}
\usepackage{pifont}
\usepackage{bbding}
\usepackage{bm}


% ---------- * 图 * 形 * 相 * 关 * -------------
\usepackage{graphicx}         % 请在引用图片时务必给出后缀名
\usepackage[x11names]{xcolor} % 支持彩色
\usepackage[below]{placeins}  % 浮动图形控制宏包
\usepackage{rotating}	      % 图形和表格的控制
\usepackage{picinpar}
\usepackage{setspace}         % 定制表格和图形的多行标题行距
\usepackage{subfigure}           % 插入子图形
\usepackage[subfigure]{ccaption} % 插图表格的双语标题
\usepackage{overpic}

% --------------- * 含 * 框 * 定 * 理 * 环 * 境 * 定 * 义 * 用 * 宏 * 包 * --------------- 
\usepackage[framemethod=TikZ]{mdframed}

\usepackage{listings}         % 源代码展示

% --------- * 大 * 工 * 原 * 始 * 设 * 置 * ---------
%\lstset{%
%  language=matlab,
%  defaultdialect=empty,
%  basicstyle=\ttfamily\small,
%  backgroundcolor=\color{LightSteelBlue1},
%  keywordstyle=\color{blue},
%  showspaces=false,
%  showstringspaces=false,
%  showtabs=false,
%  tabsize=2,breakatwhitespace=false,
%  columns=flexible}

% ---------- * 自 * 定 * 义 * 代 * 码 * 格 * 式 * 设 * 置 * ---------
\renewcommand{\lstlistlistingname}{程\quad 序}
\renewcommand{\lstlistingname}{例}
\contentsuse{lstlisting}{lol}
\titlecontents{lstlisting}[0pt]{\vspace{0.05\baselineskip}}{}{\thecontentslabel\quad}{\hspace{.5em}\titlerule*[8pt]{$\cdot$}\contentspage}
\lstset{
	language=matlab,
	%basicstyle=\normalsize\tt,
	basicstyle=\small\tt,
	frame=l,
	numbers=left,
	numberstyle=\footnotesize,
	showstringspaces=false,
	%breaklines=true,
	breaklines=false,
	breakatwhitespace=false,
}
\usepackage{caption}
\captionsetup[lstlisting]{labelformat=empty,labelsep=none,format=plain,font=bf}


%\usepackage{verbatim}
% 其他
\usepackage{calc}   % 在 tex 文件中具有一些计算功能,主要用在页面控制。
%\usepackage[xetex,
%bookmarksnumbered=true,
%bookmarksopen=true,
%colorlinks=true,
%% pdfborder={0 0 1},
%citecolor=blue,
%linkcolor=blue,
%anchorcolor=green,
%urlcolor=magenta,
%breaklinks=true,
%CJKbookmarks=true,
%]{hyperref}

% ----------------------- MATLAB * 代 * 码 * 格 * 式 * 设 * 置 ------------------------
%\usepackage[minted]{tcolorbox}
%\tcbuselibrary{skins, listings, xparse, breakable}
\usepackage[framed,numbered,autolinebreaks,useliterate]{mcode}

% ---------------------- * 超 * 链 * 接 * 与 * 书 * 签 * 设 * 置 * -----------------------------
\usepackage[CJKbookmarks,
colorlinks=true,
bookmarksnumbered=true,%
pdfstartview=FitH,
linkcolor=blue,
anchorcolor=violet,
citecolor=magenta]{hyperref}  %书签功能,选项去掉链接红色方框

% --------------------- 超 * 链 * 接 * 自 * 动 * 断 * 行 * 设 * 置 * -----------------------
\makeatletter
\def\UrlAlphabet{%
	\do\a\do\b\do\c\do\d\do\e\do\f\do\g\do\h\do\i\do\j%
	\do\k\do\l\do\m\do\n\do\o\do\p\do\q\do\r\do\s\do\t%
	\do\u\do\v\do\w\do\x\do\y\do\z\do\A\do\B\do\C\do\D%
	\do\E\do\F\do\G\do\H\do\I\do\J\do\K\do\L\do\M\do\N%
	\do\O\do\P\do\Q\do\R\do\S\do\T\do\U\do\V\do\W\do\X%
	\do\Y\do\Z}
\def\UrlDigits{\do\1\do\2\do\3\do\4\do\5\do\6\do\7\do\8\do\9\do\0}
\g@addto@macro{\UrlBreaks}{\UrlOrds}
\g@addto@macro{\UrlBreaks}{\UrlAlphabet}
\g@addto@macro{\UrlBreaks}{\UrlDigits}
\makeatother

\usepackage[backend=biber,style=gb7714_2015,]{biblatex}
\setlength{\bibitemsep}{2pt}
%\defbibheading{bibliography}[\bibname]{%
%	\phantomsection%解决链接指引出错的问题,相当于加入了一个引导点
%\addcontentsline{toc}{subsubsection}{#1}
%\centering\subsubsection*{#1}}%
\defbibheading{bibliography}[\bibname]{%
	\chapter*{#1}%
	\markboth{#1}{#1}}
% ------------- * BibLaTeX * 文 * 献 * 设 * 置 * ---------------
%\addbibresource[location=local]{PHDReferences.bib}
\addbibresource[location=local]{Ref.bib}
%\usepackage{ulem}
%\newcommand\yellowback{\bgroup\markoverwith
%	{\textcolor{yellow}{\rule[-0.5ex]{2pt}{2.5ex}}}\ULon}
%\newcommand\reduline{\bgroup\markoverwith
%	{\textcolor{red}{\rule[-0.5ex]{2pt}{0.4pt}}}\ULon}

% ---------- * 文 * 件 * 夹 * 与 * 路 * 径 * logo * 显 * 示 * ------------
\usepackage{menukeys}
\usepackage{hologo}
\renewmenumacro{\directory}{pathswithfolder}
