% !Mode:: "TeX:UTF-8"
% !TEX TS-program = XeLaTeX
% !TEX encoding = UTF-8 Unicode

% ==================================================================
%                                                                 %                                                       
%	东北大学博士论文 XeLaTeX 模版 —— 格式文件 format.tex           %
%	版本:0.22                                                  %
%	最后更新:2017.06.25                                         %
%   	二次修订:bainhome (E-main: maliang7653@sina.com)            %
%   	编译环境:Windows 7 SP1 + TeXLive 2016 + TeXStudio2.12.2     %
%                                                                  % 
% ==================================================================

% 页面设置——A4 纸张
\setlength{\paperwidth}{21.0cm}
\setlength{\paperheight}{29.7cm}
% 设置正文尺寸大小
\setlength{\textwidth}{16.0cm}
\setlength{\textheight}{22.7cm}
% 设置正文区在正中间
\newlength \mymargin
\setlength{\mymargin}{(\paperwidth-\textwidth)/2}
\setlength{\oddsidemargin}{(\mymargin)-1in}
\setlength{\evensidemargin}{(\mymargin)-1in}
% 设置正文区偏移量,奇数页向右偏,偶数页向左偏
\newlength \myshift
\setlength{\myshift}{0.35cm}	% 双面打印的奇偶页偏移值,可根据需要修改,建议小于 0.5cm
\addtolength{\oddsidemargin}{\myshift}
\addtolength{\evensidemargin}{-\myshift}
% 页眉页脚相关距离设置
\setlength{\topmargin}{-0.05cm}
\setlength{\headheight}{0.50cm}
\setlength{\headsep}{0.60cm}
\setlength{\footskip}{1.3cm}
% 公式的精调
\allowdisplaybreaks[4]  % 可以让公式在排不下的时候分页排,这可避免页面有大段空白。

% 下面这组命令使浮动对象的缺省值稍微宽松一点,从而防止幅度
% 对象占据过多的文本页面,也可以防止在很大空白的浮动页上放置很小的图形。
\renewcommand{\topfraction}{0.9999999}
\renewcommand{\textfraction}{0.0000001}
\renewcommand{\floatpagefraction}{0.9999}

% ---------- * 公 * 式 * 测 * 试 * 命 * 令 * 定 * 义 * ---------------
\newcommand{\TEST}[1]{\[#1\] \[2^{#1}\] \[2^{2^{#1}}\]}

% ---------- * 字 * 体 * 字 * 号 * 定 * 义 * -----------------

% 字号
\newcommand{\yihao}{\fontsize{26pt}{39pt}\selectfont}	    % 一号,1.5  倍行距
\newcommand{\xiaoyi}{\fontsize{24pt}{30pt}\selectfont}      % 小一,1.25 倍行距
\newcommand{\erhao}{\fontsize{22pt}{44pt}\selectfont}     % 二号,2.0 倍行距
\newcommand{\xiaoer}{\fontsize{18pt}{22.5pt}\selectfont}    % 小二,1.25 倍行距
\newcommand{\sanhao}{\fontsize{16pt}{24pt}\selectfont}      % 三号,1.5 倍行距
\newcommand{\xiaosan}{\fontsize{15pt}{19pt}\selectfont}     % 小三,1.25 倍行距
\newcommand{\sihao}{\fontsize{14pt}{21pt}\selectfont}     % 四号,1.25倍行距
\newcommand{\daxiaosi}{\fontsize{12pt}{18pt}\selectfont}    % 小四,1.5 倍行距
\newcommand{\xiaosi}{\fontsize{12pt}{15pt}\selectfont}      % 小四,1.25倍行距
\newcommand{\dawu}{\fontsize{10.5pt}{18pt}\selectfont}      % 五号,1.75倍行距
\newcommand{\zhongwu}{\fontsize{10.5pt}{16pt}\selectfont}   % 五号,1.5 倍行距
\newcommand{\wuhao}{\fontsize{10.5pt}{10.5pt}\selectfont}   % 五号,单倍行距
\newcommand{\xiaowu}{\fontsize{9pt}{9pt}\selectfont}	    % 小五,单倍行距

\newcommand{\song}{\CJKfamily{song}}
\newcommand{\hei}{\CJKfamily{hei}}
\newcommand{\kai}{\CJKfamily{kai}}
\newcommand{\fs}{\CJKfamily{fs}}
\newcommand{\xkai}{\CJKfamily{xkai}}
\newcommand{\ag}{\CJKfamily{agaramond}}
\newcommand{\minion}{\CJKfamily{minionpro}}

% defaultfont 默认字体命令
\def\defaultfont{\renewcommand{\baselinestretch}{1.27}
  \fontsize{12pt}{15pt}\selectfont}

% 设置目录字体和行间距
\def\defaultmenufont{\renewcommand{\baselinestretch}{1.22}
  \fontsize{12pt}{15pt}\selectfont}

% 固定距离内容填入及下划线
\makeatletter
\newcommand\fixeddistanceleft[2][1cm]{{\hb@xt@ #1{#2\hss}}}
\newcommand\fixeddistancecenter[2][1cm]{{\hb@xt@ #1{\hss#2\hss}}}
\newcommand\fixeddistanceright[2][1cm]{{\hb@xt@ #1{\hss#2}}}
\newcommand\fixedunderlineleft[2][1cm]{\underline{\hb@xt@ #1{#2\hss}}}
\newcommand\fixedunderlinecenter[2][1cm]{\underline{\hb@xt@ #1{\hss#2\hss}}}
\newcommand\fixedunderlineright[2][1cm]{\underline{\hb@xt@ #1{\hss#2}}}
\makeatother

% ---------------- * 标 * 题 * 环 * 境 * 相 * 关 *  ------------------- 
% 定义、定理等环境
\theoremstyle{plain}
\theoremheaderfont{\hei\bf}
\theorembodyfont{\song\rmfamily}
\newtheorem{definition}{\hei 定义}[chapter]
\newtheorem{example}{\hei 例}[chapter]
\newtheorem{algorithm}{\hei 算法}[chapter]
\newtheorem{theorem}{\hei 定理}[chapter]
\newtheorem{axiom}{\hei 公理}[chapter]
\newtheorem{proposition}[theorem]{\hei 命题}
\newtheorem{property}{\hei 性质}
\newtheorem{lemma}[theorem]{\hei 引理}
\newtheorem{corollary}{\hei 推论}[chapter]
\newtheorem{remark}{\hei 注解}[chapter]
%\newenvironment{proof}{\hei{证明} }{\hfill $\square$ \vskip 4mm}

% ------------- * 彩 * 色 * 含 * 框 * 定 * 理 * 环 * 境 * 定 * 义 *  ------------ 
%Theorem
\newcounter{theo}[section] \setcounter{theo}{0}
\renewcommand{\thetheo}{\arabic{section}.\arabic{theo}}
\newenvironment{theo}[2][]{%
	\refstepcounter{theo}%
	\ifstrempty{#1}%
	{\mdfsetup{%
			frametitle={%
				\tikz[baseline=(current bounding box.east),outer sep=0pt]
				\node[anchor=east,rectangle,fill=blue!20]
				{\strut Theorem~\thetheo};}}
	}%
	{\mdfsetup{%
			frametitle={%
				\tikz[baseline=(current bounding box.east),outer sep=0pt]
				\node[anchor=east,rectangle,fill=blue!20]
				{\strut Theorem~\thetheo:~#1};}}%
	}%
	\mdfsetup{innertopmargin=10pt,linecolor=blue!20,%
		linewidth=2pt,topline=true,%
		frametitleaboveskip=\dimexpr-\ht\strutbox\relax
	}
	\begin{mdframed}[]\relax%
		\label{#2}}{\end{mdframed}}
%%%%%%%%%%%%%%%%%%%%%%%%%%%%%%
%Lemma
\newcounter{lem}[section] \setcounter{lem}{0}
\renewcommand{\thelem}{\arabic{section}.\arabic{lem}}
\newenvironment{lem}[2][]{%
	\refstepcounter{lem}%
	\ifstrempty{#1}%
	{\mdfsetup{%
			frametitle={%
				\tikz[baseline=(current bounding box.east),outer sep=0pt]
				\node[anchor=east,rectangle,fill=green!20]
				{\strut Lemma~\thelem};}}
	}%
	{\mdfsetup{%
			frametitle={%
				\tikz[baseline=(current bounding box.east),outer sep=0pt]
				\node[anchor=east,rectangle,fill=green!20]
				{\strut Lemma~\thetheo:~#1};}}%
	}%
	\mdfsetup{innertopmargin=10pt,linecolor=green!20,%
		linewidth=2pt,topline=true,%
		frametitleaboveskip=\dimexpr-\ht\strutbox\relax
	}
	\begin{mdframed}[]\relax%
		\label{#2}}{\end{mdframed}}
%%%%%%%%%%%%%%%%%%%%%%%%%%%%%%
%Proof
\newcounter{prf}[section]\setcounter{prf}{0}
\renewcommand{\theprf}{\arabic{section}.\arabic{prf}}
\newenvironment{prf}[2][]{%
	\refstepcounter{prf}%
	\ifstrempty{#1}%
	{\mdfsetup{%
			frametitle={%
				\tikz[baseline=(current bounding box.east),outer sep=0pt]
				\node[anchor=east,rectangle,fill=red!20]
				{\strut Proof~\theprf};}}
	}%
	{\mdfsetup{%
			frametitle={%
				\tikz[baseline=(current bounding box.east),outer sep=0pt]
				\node[anchor=east,rectangle,fill=red!20]
				{\strut Proof~\thetheo:~#1};}}%
	}%
	\mdfsetup{innertopmargin=10pt,linecolor=red!20,%
		linewidth=2pt,topline=true,%
		frametitleaboveskip=\dimexpr-\ht\strutbox\relax
	}
	\begin{mdframed}[]\relax%
		\label{#2}}{\qed\end{mdframed}}

% 目录标题
\renewcommand\contentsname{\hfill 目  录 \hfill}
\renewcommand\listfigurename{\hfill 插~图~目~录 \hfill}
\renewcommand\listtablename{\hfill 表~格~目~录 \hfill}
\renewcommand{\bibname}{\hfill 参~考~文~献 \hfill}

%%%%%%%%%%%%%%%%%%%%%%%%%%%%%%%%%%%%%%%%%%%%%%%%%%%%%%%%%%%%%%%%%%%%%%
% 段落章节相关
%%%%%%%%%%%%%%%%%%%%%%%%%%%%%%%%%%%%%%%%%%%%%%%%%%%%%%%%%%%%%%%%%%%%%%
\setcounter{secnumdepth}{4}
\setcounter{tocdepth}{4}
% 设置章、节、小节、小小节的间距
\titleformat{\chapter}[hang]{\normalfont\filcenter\erhao\hei \sf}{\hei\erhao 第\erhao\thechapter\hei\erhao 章}{1em}{\hei\erhao}
%\titleformat{\chapter*}[display]{\erhao\hei}{\hei\erhao}
\titlespacing{\chapter}{0pt}{-3ex  plus .1ex minus .2ex}{3.0ex}
\titleformat{\section}[hang]{\sanhao\hei\sf}{\sanhao\thesection}{1em}{\hei}{}
\titlespacing{\section}{0pt}{0.5em}{0.5em}
\titleformat{\subsection}[hang]{\sihao\hei\sf}{\sihao\thesubsection}{1em}{\hei}{}
\titlespacing{\subsection}{0pt}{0.5em}{0.3em}
\titleformat{\subsubsection}[hang]{\hei\sf}{\thesubsubsection}{1em}{}{}
\titlespacing{\subsubsection}{0pt}{0.3em}{0pt}
% 缩小目录中各级标题之间的缩进
\dottedcontents{chapter}[0.32cm]{\vspace{0.2em}}{1.0em}{5pt}
\dottedcontents{section}[1.32cm]{}{1.8em}{5pt}
\dottedcontents{subsection}[2.32cm]{}{2.7em}{5pt}
\dottedcontents{subsubsection}[3.32cm]{}{3.4em}{5pt}

% 段落之间的竖直距离
\setlength{\parskip}{1.2pt}
% 段落缩进
\setlength{\parindent}{24pt}
% 定义行距
\renewcommand{\baselinestretch}{1.27}
% 参考文献条目间行间距
%\setlength{\bibsep}{2pt}

%%%%%%%%%%%%%%%%%%%%%%%%%%%%%%%%%%%%%%%%%%%%%%%%%%%%%%%%%%%%%%%%%%%%%%
% 页眉页脚设置
%%%%%%%%%%%%%%%%%%%%%%%%%%%%%%%%%%%%%%%%%%%%%%%%%%%%%%%%%%%%%%%%%%%%%%

\newcommand{\makeheadrule}{%
  \makebox[0pt][l]{\rule[.7\baselineskip]{\headwidth}{0.5pt}}%
  \vskip-.8\baselineskip}

\makeatletter
\renewcommand{\headrule}{%
  {\if@fancyplain\let\headrulewidth\plainheadrulewidth\fi
    \makeheadrule}}

\pagestyle{fancyplain}

\fancyhf{}
\fancyhead[L]{\@topNEU}
\fancyhead[R]{\kai\wuhao\leftmark}
\fancyfoot[C,C]{\xiaowu$-$~\thepage~$-$}

% ---------- * 定 * 义 * 页 * 眉 * 中 * 文 * 章 * 标 * 题 * 格 * 式 * ----------
\renewcommand{\chaptermark}[1]{\markboth{第\arabic{chapter}章~#1 }{}}
\renewcommand{\sectionmark}[1]{\markright{ \S\arabic{chapter}-\arabic{section}\  #1}{}}

% Clear Header Style on the Last Empty Odd pages
\makeatletter
\def\cleardoublepage{\clearpage\if@twoside \ifodd\c@page\else%
  \hbox{}%
  \thispagestyle{empty}%              % Empty header styles
  \newpage%
  \if@twocolumn\hbox{}\newpage\fi\fi\fi}



%%%%%%%%%%%%%%%%%%%%%%%%%%%%%%%%%%%%%%%%%%%%%%%%%%%%%%%%%%%%%%%%%%%%%%
% 列表环境设置

%%%%%%%%%%%%%%%%%%%%%%%%%%%%%%%%%%%%%%%%%%%%%%%%%%%%%%%%%%%%%%%%%%%%%%

\setlist[enumerate]{(1),itemsep=-5pt,topsep=0mm,labelindent=\parindent,leftmargin=*}


%%%%%%%%%%%%%%%%%%%%%%%%%%%%%%%%%%%%%%%%%%%%%%%%%%%%%%%%%%%%%%%%%%%%%%
% 国际单位,以点连接。
%%%%%%%%%%%%%%%%%%%%%%%%%%%%%%%%%%%%%%%%%%%%%%%%%%%%%%%%%%%%%%%%%%%%%%
\sisetup{inter-unit-product = { }\cdot{ }}

%%%%%%%%%%%%%%%%%%%%%%%%%%%%%%%%%%%%%%%%%%%%%%%%%%%%%%%%%%%%%%%%%%%%%%
% 参考文献的处理
%%%%%%%%%%%%%%%%%%%%%%%%%%%%%%%%%%%%%%%%%%%%%%%%%%%%%%%%%%%%%%%%%%%%%%

% \addtolength{\bibsep}{-0.5em}              % 缩小参考文献间的垂直间距

%\setlength{\bibhang}{2em}
%\bibpunct{[}{]}{,}{s}{}{}

%简化偏导数的输入
%---------------------------------------------
%使用示范:\piandao{a}{b} 表示 a 对 b 的 1 阶偏导数
%或者 \piandao[n]{a}{b^n} 表示 a 对 b 的 n 阶偏导
\newcommand{\piandao}[3][1]{
	\ifthenelse{\equal{#1}{1}}{ %默认 1 阶导数
		\frac{\partial #2}{\partial #3}
	}{
		\frac{\partial^{#1} #2}{\partial #3}
	}
}

%定义数学直体的微分算子
\newcommand*{\opd}{\mathop{}\!\mathrm{d}}
%\newcommand{\opD}{\mathrm{D}}

%导数,小写d
\newcommand{\daoshu}[3][1]{
	\ifthenelse{\equal{#1}{1}}{ %默认 1 阶导数
		\frac{\opd #2}{\opd #3}
	}{
		\frac{\opd^{#1} #2}{\opd #3}
	}
}


% \let\orig@Itemize =\itemize
% \let\orig@Enumerate =\enumerate
% \let\orig@Description =\description

% \def\Myspacing{\itemsep=1ex \topsep=-4ex \partopsep=-2ex \parskip=-1ex \parsep=2ex}
% \def\newitemsep{
% \renewenvironment{itemize}{\orig@Itemize\Myspacing}{\endlist}
% \renewenvironment{enumerate}{\orig@Enumerate\Myspacing}{\endlist}
% \renewenvironment{description}{\orig@Description\Myspacing}{\endlist}
% }
%   \def\olditemsep{
%   \renewenvironment{itemize}{\orig@Itemize}{\endlist}
%   \renewenvironment{enumerate}{\orig@Enumerate}{\endlist}
%   \renewenvironment{description}{\orig@Description}{\endlist}
% }
%   \renewcommand{\labelenumi}{(\arabic{enumi})}
%   \newitemsep

%%%%%%%%%%%%%%%%%%%%%%%%%%%%%%%%%%%%%%%%%%%%%%%%%%%%%%%%%%%%%%%%%%%%%%
%   其他设置
%%%%%%%%%%%%%%%%%%%%%%%%%%%%%%%%%%%%%%%%%%%%%%%%%%%%%%%%%%%%%%%%%%%%%%
%   增加 \ucite 命令使显示的引用为上标形式
%   \newcommand{\ucite}[1]{$^{\mbox{\scriptsize \cite{#1}}}$}

% -------------------------------------
%   图形表格
% -------------------------------------
\renewcommand{\figurename}{图}
\renewcommand{\tablename}{表}
% \captionstyle{\centering}
% \hangcaption
\captiondelim{\hspace{1em}}
\captiondelim{\hspace{1em}}
\captionnamefont{\zhongwu}
\captiontitlefont{\zhongwu}
\setlength{\abovecaptionskip}{0pt}
\setlength{\belowcaptionskip}{0pt}



\newcommand{\tablepage}[2]{\begin{minipage}{#1}\vspace{0.5ex} #2 \vspace{0.5ex}\end{minipage}}
\newcommand{\returnpage}[2]{\begin{minipage}{#1}\vspace{0.5ex} #2 \vspace{-1.5ex}\end{minipage}}


% -------------------------------------
% 定义题头格言的格式
% -------------------------------------

\newsavebox{\AphorismAuthor}
\newenvironment{Aphorism}[1]
{\vspace{0.5cm}\begin{sloppypar} \slshape
    \sbox{\AphorismAuthor}{#1}
    \begin{quote}\small\itshape }
    {\\ \hspace*{\fill}------\hspace{0.2cm} \usebox{\AphorismAuthor}
    \end{quote}
  \end{sloppypar}\vspace{0.5cm}}

% 自定义一个空命令,用于注释掉文本中不需要的部分。
\newcommand{\comment}[1]{}

% This is the flag for longer version
\newcommand{\longer}[2]{#1}

\newcommand{\ds}{\displaystyle}

% define graph scale
\def\gs{1.0}

% -------------------------------------
% 封面摘要
% -------------------------------------
\def\cdegree#1{\def\@cdegree{#1}}\def\@cdegree{}
\def\ctitle#1{\def\@ctitle{#1}}\def\@ctitle{}
\def\caffil#1{\def\@caffil{#1}}\def\@caffil{}
\def\csubject#1{\def\@csubject{#1}}\def\@csubject{}
\def\cauthor#1{\def\@cauthor{#1}}\def\@cauthor{}
\def\eauthor#1{\def\@eauthor{#1}}\def\@eauthor{}
\def\cauthorno#1{\def\@cauthorno{#1}}\def\@cauthorno{}
\def\csupervisor#1{\def\@csupervisor{#1}}\def\@csupervisor{}
\def\esupervisor#1{\def\@esupervisor{#1}}\def\@esupervisor{}
\def\cdate#1{\def\@cdate{#1}}\def\@cdate{}
\long\def\cabstract#1{\long\def\@cabstract{#1}}\long\def\@cabstract{}
\def\ckeywords#1{\def\@ckeywords{#1}}\def\@ckeywords{}
\def\etitle#1{\def\@etitle{#1}}\def\@etitle{}
\def\etitle#1{\def\@etitle{#1}}\def\@etitle{}
\long\def\eabstract#1{\long\def\@eabstract{#1}}\long\def\@eabstract{}
\def\ekeywords#1{\def\@ekeywords{#1}}\def\@ekeywords{}

\def\topNEU#1{\def\@topNEU{#1}}\def\@topNEU{}
\def\cNEU#1{\def\@cNEU{#1}}\def\@cNEU{}
\def\eNEU#1{\def\@eNEU{#1}}\def\@eNEU{}
\def\cMajor#1{\def\@cMajor{#1}}\def\@cMajor{}
\def\eMajor#1{\def\@eMajor{#1}}\def\@eMajor{}
\def\cDegree#1{\def\@cDegree{#1}}\def\@cDegree{}
\def\funnynameA#1{\def\@funnynameA{#1}}\def\@funnynameA{}
\def\funnynameB#1{\def\@funnynameB{#1}}\def\@funnynameB{}
% 补充定义:所属学院名称
\def\cNEUpartment#1{\def\@cNEUpartment{#1}}\def\@cNEUpartment{}
% 补充定义:年月日期
\def\tmdate#1{\def\@tmdate{#1}}\def\@tmdate{}
\def\edate#1{\def\@edate{#1}}\def\@edate{}

% -------------------- * 封 * 面 * -------------------------
\def\makecover{
  \begin{titlepage}
  % 中文封面
    \newpage
    \thispagestyle{empty}
    \sihao{\hei{
    \begin{tabular}{lccc}
        分类号  & \underline{\hspace{3.9cm}}~ & 密级 & \underline{\hspace{2.5cm}} \\
        UDC & \underline{\hspace{3.9cm}} &  &  \\
    \end{tabular}
    }}
    \vspace{1.5cm}
    \begin{center}
      \parbox[t][4.40cm][c]{\textwidth}
      {
        \begin{center}
          {\erhao\song\@cdegree\\}
          \vspace{1cm}
          {\xiaoer\hei\textsf{\@ctitle}\\}
          \vspace{2.3cm}
        \end{center}
      }
      \parbox[b][9.39cm][c]{\textwidth}
      {
        \begin{center}
          {\renewcommand{\baselinestretch}{1.61}
            \sihao\song
            \begin{tabular}{rl}
              作 \hfill 者 \hfill 姓 \hfill 名:& \@cauthor\\
              指 \hfill 导 \hfill 教 \hfill 师:& \@csupervisor \\
              ~                                 & \@cNEUpartment\\
              申 \hfill 请 \hfill 学 \hfill 位 \hfill 级 \hfill 别:& \@cDegree\hspace{1cm}学~科~类~别:\@cMajor\\
              学 \hfill 科 \hfill 专 \hfill 业 \hfill 名 \hfill 称:& \@csubject\\
              论 \hfill 文 \hfill 提 \hfill 交 \hfill 日 \hfill 期:& \@tmdate~论文答辩日期:\the\year ~ 年~~~ 月 \\
              学 \hfill 位 \hfill 授 \hfill 予 \hfill 日 \hfill 期:& \hspace{1.2cm}年\hspace{.4cm}月\hspace{.1cm}答辩委员会主席:\\
              评 \hfill 阅 \hfill 人:& \\
            \end{tabular}
          }
        \end{center}
      }
\renewcommand{\baselinestretch}{1.27}
\,\hfil\,

\,\hfil\,

\,\hfil\,

      {
     	\sihao\song
             \@cNEU\\
        \@tmdate
      }
    \end{center}
    \cleardoublepage
%------------------------ * 英 * 文 * 封 * 面 * -------------------------
    \newpage
    \thispagestyle{empty}
    \sihao{\textbf{A Dissertation in \@eMajor}}
    \vspace{1.2cm}
    \begin{center}
    	\renewcommand{\baselinestretch}{1}{
          {\erhao\textbf{\@etitle}\\}
          }
          \vspace{2.0cm}
          {By \@eauthor \\}
          \vspace{2.0cm}
          {Supervisor: \@esupervisor \\}
      \vspace{8.6cm}
     	\sanhao \textbf{\@eNEU\\}
        \vspace{0.34cm}
        \textbf{\@edate}\\
    \end{center}
\cleardoublepage
% ------ * 授 * 权 * 书 * 与 * 独 * 创 * 性 * 声 * 明 * --------
    \newpage
    \thispagestyle{empty}
    \begin{flushleft}
      {\center\erhao\hei 独创性声明\\}
        \addcontentsline{toc}{chapter}{\hei 独创性声明}
        \setcounter{page}{1}
    \vspace{0.6cm}
    {
      \sihao\song
    \hspace{0.85cm} 本人声明,所呈交的学位论文是在导师的指导下完成的。论文中取得的研究成果除加以标注和致谢的地方外,不包含其他人己经发表或撰写过的研究成果,也不包括本人为获得其他学位而使用过的材料。与我一同工作的同志对本研究所做的任何贡献均己在论文中作了明确的说明并表示谢意。\\
      \vspace{0.5cm}
    }
    {\hspace{8.3cm}学位论文作者签名:\\\vspace{0.35cm}\hspace{8.3cm}日\hspace{1.1cm}期:\\}
    \vspace{1.2cm}
    {\center\erhao\hei 学位论文版权使用授权书\\}
    \vspace{0.6cm}
    \sihao\song
     \hspace{0.85cm} 本学位论文作者和指导教师完全了解东北大学有关保留、使用学位论文的规定:即学校有权保留并向国家有关部门或机构送交论文的复印件和磁盘,允许论文被查阅和借阅。本人同意东北大学可以将学位论文的全部或部分内容编入有关数据库进行检索、交流。

        \vspace{15pt}

    \hspace{0.85cm} 作者和导师同意网上交流的时间为作者获得学位后:\\
        \vspace{15pt}
    \hspace{0.85cm} 半年~$\Box$ \hspace{1cm} 一年~$\Box$ \hspace{1cm} 一年半~$\Box$ \hspace{1cm} 两年~$\Box$
    \end{flushleft}
    \vspace{0.7cm}
\noindent
    \begin{tabular}{llll}
      学位论文作者签名: & \hspace{2.5cm} & 导师签名: & ~ \\
      签字日期: & \hfill & 签字日期: & ~ \\
    \end{tabular}
    \cleardoublepage
  \end{titlepage}
}

\def\makeabstract{
  \defaultfont
  \section*{}
  \rhead{\kai\wuhao 摘 要}
%  \chapter*{\xiaosan\hei\@ctitle}
  \centerline{\erhao\hei\hfill 摘 要 \hfill }
  \vspace{0.5cm}
  \addcontentsline{toc}{chapter}{摘 要}
  \setcounter{page}{2}
  \@cabstract
  \vspace{0.53cm}

  \noindent {\hei{关键词:{\song\@ckeywords}}}

  \defaultfont
  \cleardoublepage
  \section*{}
  \rhead{\kai\wuhao Abstract}
  \addcontentsline{toc}{chapter}{Abstract}
%  \vspace{-1.40cm}
%  \begin{center}
%    {\erhao\textrm{\@etitle}}
%  \end{center}
  \vspace{-0.35cm}
  \begin{center}
    {
      \erhao{Abstract}\\
    }
  \end{center}
  \vspace{0.12cm}

  \@eabstract

  \vspace{0.55cm}

  \noindent {\textbf{Key Words:}~~\@ekeywords}
  \cleardoublepage
}

\makeatletter
\def\hlinewd#1{%
  \noalign{\ifnum0=`}\fi\hrule \@height #1 \futurelet
  \reserved@a\@xhline}
\makeatother

% 定义索引生成
\def\generateindex
{
  \addcontentsline{toc}{chapter}{\indexname}
  \printindex
  \cleardoublepage
}

\raggedbottom
