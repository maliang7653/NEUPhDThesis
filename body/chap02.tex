% !Mode:: "TeX:UTF-8"
% !TEX TS-program = XeLaTeX
% !TEX encoding = UTF-8 Unicode

\chapter{列表、插图}
\label{chap02}

\section{列表}
\label{sec:23}

如果不是搞昆虫学,那就是连着几个月记着写那本大书《文化史上德谟克利特和柏拉
图两个流派》,或者是《形态学的发展》,或者是《应用生物学中的统计方法》,再不
然是他一九五一年至一九五二年编写的教程。几十、几百页都是这种枯燥无味、事
务性的记载,每天五至七行\footnote{格拉宁:《奇特的一生》页20}:
\begin{enumerate}
\item 鉴定袋蛾———二十分
\item 给斯拉瓦写信———二小时四十五分
\item 植物保护小组开会———二小时二十五分
\end{enumerate}

上述是默认的列表样式。源代码如下:
\begin{lstlisting}
  \begin{enumerate}
  \item 鉴定袋蛾———二十分
  \item 给斯拉瓦写信———二小时四十五分
  \item 植物保护小组开会———二小时二十五分
  \end{enumerate}
\end{lstlisting}

\texttt{emumerate}环境就是列表环境。每条\texttt{\textbackslash{item}}后面
跟一个空格,然后就是具体的条目。

默认的样式是按照(1),(2),(3)来排序的,如果想按照英文字母(a),(b),(c)或者罗
马数字(i),(ii),(iii)这样的顺序呢,只需要
在\texttt{\textbackslash{begin}\{enumerate\}}后面加一个参数,参数放在方括
号内。比如:
\begin{enumerate}[(a)]
\item 鉴定袋蛾———二十分
\item 给斯拉瓦写信———二小时四十五分
\item 植物保护小组开会———二小时二十五分
\end{enumerate}
源代码是:
\begin{lstlisting}
  \begin{enumerate}[(a)]
  \item 鉴定袋蛾———二十分
  \item 给斯拉瓦写信———二小时四十五分
  \item 植物保护小组开会———二小时二十五分
  \end{enumerate}
\end{lstlisting}

如上,方括号的中参数是可以更改的。a代表小写字母,A代表大写字母,1代表数
字,i代表小写罗马数字,I代表大写罗马数字。这些参数可以加上圆括号,也可以
加上一个点(英文句号)。\textcolor{red}{[a)]}:列表的标签就会变
成a)、b)、c)。\textcolor{red}{[1.]}:列表的标签就会变成1.、2.、3. 。

罗马数字的例子:
\begin{enumerate}[i.]
\item 鉴定袋蛾———二十分
\item 给斯拉瓦写信———二小时四十五分
\item 植物保护小组开会———二小时二十五分
\end{enumerate}
源代码:
\begin{lstlisting}
  \begin{enumerate}[i.]
  \item 鉴定袋蛾———二十分
  \item 给斯拉瓦写信———二小时四十五分
  \item 植物保护小组开会———二小时二十五分
  \end{enumerate}
\end{lstlisting}

\section{表}

\subsection{三线表}

\begin{figure}[htbp]
  \centering
  \includegraphics[width=0.9\textwidth{},keepaspectratio]{sun.jpg}
  \bicaption[fig:sun]{图}{最左侧是太阳,向右依序为水星、金星、地球、火星、
    木星、土星、天王星与海王星}{Fig.}{Outward from the Sun, the planets
    are Mercury, Venus, Earth, Mars, Jupiter, Saturn, Uranus and
    Neptune.}
\end{figure}

天文学家在太阳系内以天文单位(AU)来测量距离。1AU是地球到太阳的平均距离,
大约是\num{149598000}公里(\num{93000000}英里)。冥王星与太阳的距离大约
是39AU,木星则约是5.2AU。最常用在测量恒星距离的长度单位是光年,1光年大约
相当于\num{63240}天文单位\footnote{维基百科:太阳系}。

图~\ref{fig:sun}展示了太阳系的各行星的位置。

\begin{table}[htbp]
  \bicaption[tab:xingxing]{表}{行星数据表}{Tab.}{Planet}
  \centering
  \vspace{0.2cm}
  \zhongwu
  \begin{tabular}{cccc}
    \toprule
    Planet  & Size(Earth=1) & Weight(Earth=1) & Radius  \\
    \midrule
    Mercury & 0.056         & 0.055           & 0.3871  \\
    Venus   & 0.857         & 0.815           & 0.7233  \\
    Earth   & 1.00          & 1.000           & 1.0000  \\
    Mars    & 0.151         & 0.107           & 1.5237  \\ 
    Jupiter & 1321          & 317.832         & 5.2026  \\ 
    Saturn  & 755           & 95.16           & 9.5549  \\ 
    Uranus  & 63            & 14.54           & 19.2184 \\ 
    Neptune & 58            & 17.15           & 30.1104 \\ 
    \bottomrule
  \end{tabular}
\end{table}

表~\ref{tab:xingxing}就是最简单的三线表。源代码如下:
\begin{lstlisting}
  \begin{table}[htbp]
    \bicaption[tab:xingxing]{ 表 }{ 行星数据表 }{Tab.}{Planet}
    \centering
    \vspace{0.2cm}
    \zhongwu
    \begin{tabular}{cccc}
      \toprule
      Planet  & Size(Earth=1) & Weight(Earth=1) & Radius  \\
      \midrule
      Mercury & 0.056         & 0.055           & 0.3871  \\
      Venus   & 0.857         & 0.815           & 0.7233  \\ 
      Earth   & 1.00          & 1.000           & 1.0000  \\ 
      Mars    & 0.151         & 0.107           & 1.5237  \\ 
      Jupiter & 1321          & 317.832         & 5.2026  \\ 
      Saturn  & 755           & 95.16           & 9.5549  \\ 
      Uranus  & 63            & 14.54           & 19.2184 \\ 
      Neptune & 58            & 17.15           & 30.1104 \\
      \bottomrule
    \end{tabular}
  \end{table}
\end{lstlisting}


表格和插图通常需要占据大块空间,所以在文字处理软件中用户经常需要调整它们的
位置。\texttt{table}环境可以自动完成这样的任务;这种自动调整位置的环境称作
浮动环境 (float),下一节里还会介绍插图浮动环境\footnote{包太雷:\LaTeX{}
  NOTES———雷太赫排版系统简介}。

\texttt{htbp} 选项用来指定表格的理想位置,这几个字母分别代表 here, top,
bottom,float page,也就是就这里、页顶、页尾、浮动页 (专门放浮动环境的单独
页面)。我们可以使用这几个字母的任意组合,四个字母都写上表示放哪里都无所
谓;一般不推荐单独使用h,因为\LaTeX{}自以为它的排版算法是最完美的,不愿意
被束缚手脚。

\texttt{\textbackslash{centering}} 用来使表格居
中;\texttt{\textbackslash{bicaption}} 命令设置表格标题,\LaTeX{}会自动给
浮动环境的标题加上编号。

它的官方使用说明为:
\begin{lstlisting}
  \bicaption[label]{ 中文短标题 }{ 中文标题 }{Tab.}{ 英文标题 }
\end{lstlisting}
可选参数~\texttt{\footnotesize label}~用来作为交叉引用链接。例如
表~\ref{tab:xingxing}中的\texttt{lable}为\texttt{tab:xingxing}。这里的标
签一般为英文。中文短标题一般没什么用,可以随意填。最简单就是“表”。

在表格环境中,标题必须位于表格的上方。而在图片环境中,标题的位置必须位于
图片的下方。

\texttt{tabular} 环境提供了最简单的表格功能。它用 \texttt{\&} 来分列,
用 \texttt{\textbackslash{\textbackslash{}}} 来换行;每列可以采用居中、居
左、居右等横向对齐方式,分别用 \texttt{l、c、r} 来表示。

三线表的三条横线就分别
用 \texttt{\textbackslash{toprule}}、\texttt{\textbackslash{midrule}}、
\texttt{\textbackslash{bottomrule}} 等命令表示。

\texttt{\textbackslash{vspace}\{0.2cm\}}是用来控制表格标题与表格正文的
垂直间距的,请在插入表格时务必添加。\texttt{\textbackslash{zhongwu}}是
用来调整表格内容的行距的。


\subsection{表格的列按小数点对齐}

以表~\ref{tab:xingxing}为例,想把其中的第三列按小数点对齐\footnote{参见宏
  包siunitx}。先看一下效果:

在表~\ref{tab:xiaoshu}中,我们调整了原来四列数的对齐方式。原来
是\texttt{cccc},现在是\texttt{lcSr}。第一列左对齐,第二列不变,还是居中
对齐,第四列右对齐。值得注意的是第三列,这里新引入了一个参数\texttt{S},
含义就是这一列的数字按照小数点对齐。一定是大写的S。另外,第三列的列
头Weight(Earth=1)两边也加上了大括号,因为这不是数字。在使用参
数\texttt{S}的时候,不是数字的行需要用大括号括起来,不然会造成编译错误。

\begin{table}[htbp]
  \bicaption[tab:xiaoshu]{表}{行星数据表}{Tab.}{Planet}
  \centering
  \vspace{0.2cm}
  \zhongwu
  \begin{tabular}{lcSr}
    \toprule
    Planet  & Size(Earth=1) & {Weight(Earth=1)} & Radius  \\
    \midrule
    Mercury & 0.056         & 0.055           & 0.3871  \\
    Venus   & 0.857         & 0.815           & 0.7233  \\ 
    Earth   & 1.00          & 1.000           & 1.0000  \\ 
    Mars    & 0.151         & 0.107           & 1.5237  \\ 
    Jupiter & 1321          & 317.832         & 5.2026  \\ 
    Saturn  & 755           & 95.16           & 9.5549  \\ 
    Uranus  & 63            & 14.54           & 19.2184 \\ 
    Neptune & 58            & 17.15           & 30.1104 \\ 
    \bottomrule
  \end{tabular}
\end{table}

\begin{lstlisting}
  \begin{table}[htbp]
    \bicaption[tab:xiaoshu]{ 表 }{ 行星数据表 }{Tab.}{Planet}
    \centering
    \vspace{0.2cm}
    \zhongwu
    \begin{tabular}{lcSr}
      \toprule
      Planet  & Size(Earth=1) & {Weight(Earth=1)} & Radius  \\
      \midrule
      Mercury & 0.056         & 0.055           & 0.3871  \\
      Venus   & 0.857         & 0.815           & 0.7233  \\ 
      Earth   & 1.00          & 1.000           & 1.0000  \\ 
      Mars    & 0.151         & 0.107           & 1.5237  \\ 
      Jupiter & 1321          & 317.832         & 5.2026  \\ 
      Saturn  & 755           & 95.16           & 9.5549  \\ 
      Uranus  & 63            & 14.54           & 19.2184 \\ 
      Neptune & 58            & 17.15           & 30.1104 \\ 
      \bottomrule
    \end{tabular}
  \end{table}
\end{lstlisting}

\subsection{多列三线表}

在三线表中,有些列的列头会横跨好几列的数据。一般使
用\texttt{multicolumn}命令。它的用法是:
\begin{lstlisting}
  \multicolumn{ 列数}{ 对齐方式 }{ 表格内容 }
\end{lstlisting}

“列数”是指这一列横跨的列数,在表~\ref{tab:linux}是2列,就填“2”;“对
齐方式”从\texttt{lcr}三者中选其一即可,在表~\ref{tab:linux}中是c。“表格
内容”填入自己的内容。一般还会在这一列的下面画一小横线,已示辨识。使
用\texttt{cmidrule}命令。在表~\ref{tab:linux}中,由于横跨的是第2列和第3列,
因此\texttt{cmidrule}的参数是2-3。

\begin{table}[htbp]
  \bicaption[tab:linux]{}{ 不同操作系统下的\LaTeX{} }{Tab.}{OS with \LaTeX{}}
  \centering
  \vspace{0.2cm}
  \zhongwu
  \begin{tabular}{ccc}
    \toprule
    Test    & \multicolumn{2}{c}{Common Tools} \\
    \cmidrule{2-3}
    OS         & Distribution & Editor            \\
    \midrule
    Windows    & MikTeX       & TexMakerX         \\
    Mac OS     & MacTeX       & TeXShop           \\
    Linux/Unix & TeX Live     & TeXworks          \\
    \bottomrule
  \end{tabular}
\end{table}


\begin{lstlisting}
  \begin{table}[htbp]
    \bicaption[tab:linux]{}{ 不同操作系统下的\LaTeX{}}{Tab.}{OS with \LaTeX{}}
    \centering
    \vspace{0.2cm}
    \zhongwu
    \begin{tabular}{ccc}
      \toprule
      Test    & \multicolumn{2}{c}{Common Tools} \\
      \cmidrule{2-3}
      OS         & Distribution & Editor            \\
      \midrule
      Windows    & MikTeX       & TexMakerX         \\
      Mac OS     & MacTeX       & TeXShop           \\
      Linux/Unix & TeX Live     & TeXworks          \\
      \bottomrule
    \end{tabular}
  \end{table}
\end{lstlisting}

\subsection{多行三线表}

既然有多列三线表,多行三线表也是用类似的方法解决。我们把表~\ref{tab:linux} 来改造一下,相对应的,一般使用\texttt{multirow}命令。它的用法
是:
\begin{lstlisting}
  \multirow{ 行数 }*{ 表格内容 }
\end{lstlisting}

“行数”是指竖向跨的行数,在表~\ref{tab:unix}中是2行,中间有个星号,表示自然宽度。

\begin{table}[htbp]
  \bicaption[tab:unix]{}{ 不同操作系统下的\LaTeX{} }{Tab.}{OS with \LaTeX{}}
  \centering
  \vspace{0.2cm}
  \zhongwu
  \begin{tabular}{ccc}
    \toprule
    \multirow{2}*{OS} & \multicolumn{2}{c}{Common Tools} \\
    \cmidrule{2-3}
    & Distribution & Editor            \\
    \midrule
    Windows          & MikTeX       & TexMakerX         \\
    Mac OS           & MacTeX       & TeXShop           \\
    Linux/Unix       & TeX Live     & TeXworks          \\
    \bottomrule
  \end{tabular}
\end{table}


\begin{lstlisting}
  \begin{table}[htbp]
    \bicaption[tab:unix]{}{ 不同操作系统下的\LaTeX{} }{Tab.}{OS with \LaTeX{}}
    \centering
    \vspace{0.2cm}
    \zhongwu
    \begin{tabular}{ccc}
      \toprule
      \multirow{2}*{OS} & \multicolumn{2}{c}{Common Tools} \\
      \cmidrule{2-3}
      & Distribution & Editor            \\
      \midrule
      Windows          & MikTeX       & TexMakerX         \\
      Mac OS           & MacTeX       & TeXShop           \\
      Linux/Unix       & TeX Live     & TeXworks          \\
      \bottomrule
    \end{tabular}
  \end{table}
\end{lstlisting}

\subsection{宽度控制}

有时候表格中的某行太长了,需要折行。可以使用\texttt{tabularx} 宏包的同名
环境,其语法如下:

\begin{lstlisting}
  \begin{tabularx}{ 表格总宽度 }{ 对齐方式 }
    ...
  \end{tabularx}
\end{lstlisting}

“表格总宽度”最好用\texttt{textwidth}乘以某个系数表示。例
如\texttt{0.8\textbackslash{textwidth}}表示表格宽度是版芯宽度的0.8倍。这
样出来的效果比较好看。对齐方式除了原有的\texttt{l,c,r}之外,多了一
个\texttt{X},表示某列可以折行。

\begin{table}[htbp]
  \centering
  \bicaption[tab:wall]{ 表 }{ 墙上的44句话 }{Tab.}{Mikko Kuorinki}
  \vspace{0.2cm}
  \zhongwu
  \begin{tabularx}{0.8\textwidth{}}{lX}
    \toprule
    People & Says \\
    \midrule
    Elias Canetti & If you were alone, you would cut yourself in two, so
    that one part would shape the other.\\
    Franz Kafka & In the struggle between yourself and the world,
    second the world.\\
    \bottomrule
  \end{tabularx}
\end{table}

\begin{lstlisting}
  \begin{table}[htbp]
    \centering
    \bicaption[tab:figure]{ 表 }{ 墙上的44句话 }{Tab.}{Mikko Kuorinki}
    \vspace{0.2cm}
    \zhongwu
    \begin{tabularx}{0.8\textwidth{}}{lX}
      \toprule
      People & Says \\
      \midrule Elias Canetti & If you were alone, you would cut yourself
      in two, so  that one part would shape the other.\\
      Franz Kafka & In the struggle between yourself and the world,
      second the world.\\
      \bottomrule
    \end{tabularx}
  \end{table}
\end{lstlisting}



\subsection{斜线表头}

还是有些童鞋的表示三线表不实用啊,非要回归到原来的斜线表头去。我们可以使
用宏包\texttt{diagbox}提供的命令轻松完成。不过呢,出来的表格很ugly罢了。

\texttt{diagbox}是宏包提供的主要命令。它可以带有两个必选参数,表示要生成斜
线表头的两部分内容。默认斜线是从西北到东南方向的。

需要注意的是,使用斜线表格后就不能使用三线表的三条横线,不然请看
表~\ref{tab:diagbox}的下场。正确的做法是使用最原始的\texttt{hline},见
表~\ref{tab:xiexian}。

\begin{table}[htbp]
  \bicaption[tab:diagbox]{表}{斜线表头}{Tab.}{Diagbox}
  \centering
  \vspace{0.2cm}
  \zhongwu
  \begin{tabular}{|l|ccc|}
    \toprule
    \diagbox{Times}{Day} & Mon  & Tue  & Wed  \\
    \midrule
    Morning              & used & used &      \\
    Afternoon            &      & used & used \\
    \bottomrule
  \end{tabular}
\end{table}

\begin{lstlisting}
  \begin{table}[htbp]
    \bicaption[tab:diagbox]{ 表 }{ 斜线表头 }{Tab.}{Diagbox}
    \centering
    \vspace{0.2cm}
    \zhongwu
    \begin{tabular}{|l|ccc|}
      \toprule
      \diagbox{Times}{Day} & Mon  & Tue  & Wed  \\
      \midrule
      Morning              & used & used &      \\
      Afternoon            &      & used & used \\
      \bottomrule
    \end{tabular}
  \end{table}
\end{lstlisting}

\begin{table}[htbp]
  \bicaption[tab:xiexian]{ 表 }{斜线表头}{Tab.}{Diagbox}
  \centering
  \vspace{0.2cm}
  \zhongwu
  \begin{tabular}{|l|ccc|}
    \hline
    \diagbox{Times}{Day} & Mon  & Tue  & Wed  \\
    \hline
    Morning              & used & used &      \\
    Afternoon            &      & used & used \\
    \hline
  \end{tabular}
\end{table}

\begin{lstlisting}
  \begin{table}[htbp]
    \bicaption[tab:xiexian]{ 表 }{ 斜线表头 }{Tab.}{Diagbox}
    \centering
    \vspace{0.2cm}
    \zhongwu
    \begin{tabular}{|l|ccc|}
      \hline
      \diagbox{Times}{Day} & Mon  & Tue  & Wed  \\
      \hline
      Morning              & used & used &      \\
      Afternoon            &      & used & used \\
      \hline
    \end{tabular}
  \end{table}
\end{lstlisting}


\section{图}
\label{chap02:figure}

我来北京十一年,上学,上班,也积累了不少同学同事,但我一次他们的婚礼都没
有参加过。刚毕业的时候这种邀请很多,好象是种翻天覆地日新月异的见证,不怕
那些同学们伤心,我也知道人家是真心邀请的,但我真觉得我没跟他们谁好到真有
必要参加那些婚礼,所以我就不去。这次黄总的婚礼通知的突然,但我却很想去见
证一下,我的罪恶太多,正好也让天主顺便宽恕宽恕我,沾沾喜气\footnote{蚌病
  生珠:阳光下的婚礼}。

我挺佩服黄总他们俩的,他们就真的仅仅在婚礼当天才拍所谓的“婚纱照”。我觉
得这样挺好。身为摄影部的美女,周围全是靠摄影吃饭的人,的确没必要出去花冤
枉钱就为了拍几张照片。

\begin{figure}[htbp]
  \centering
  \includegraphics[scale=0.6]{wedding.jpg}
  \bicaption[fig:wedding]{婚礼}{婚礼}{Fig.}{Wedding}
\end{figure}

\begin{lstlisting}
  \begin{figure}[htbp]
    \centering
    \includegraphics[scale=0.6]{wedding.jpg}
    \bicaption[fig:wedding]{ 婚礼 }{ 婚礼 }{Fig.}{Wedding}
  \end{figure}
\end{lstlisting}

论文使用的图片都放在figure文件夹中,插图浮动环境是\texttt{figure},基本命
令是\texttt{includegraphics},而在图片环境中,标题的位置必须位于图片的下
方。

\texttt{includegraphics}的基本参数见表~\ref{tab:figure}。

\begin{table}[htbp]
  \centering
  \bicaption[tab:figure]{插图命令参数}{插图命令参数}{Tab.}{Parameter}
  \vspace{0.2cm}
  \zhongwu
  \begin{tabularx}{0.8\textwidth{}}{lX}
    \toprule
    参数             & 说明 \\
    \midrule
    width=x,height=y & 宽度和高度,绝对尺寸,可用任意长度单位。                           \\
    scale=s          & 缩放比。绝对尺寸和缩放比用一种即可,同时使用两者,绝对尺寸起作用。 \\
    keepaspectratio & 保持图形比例。宽度和高度通常设置一个即可,否则图形比
    例会失调,除非再加上此选 项,
    这样图形宽度和高度都不超过指定参数。                                                \\
    angle=a          & 逆时针旋转角度,单位是度。                                        \\
    \bottomrule
  \end{tabularx}
\end{table}

对于图~\ref{fig:wedding},只使用了\texttt{scale}这一个参数,缩放因子是0.6。
当然,也可以直接指定图形的宽度和高度。图~\ref{fig:sun}的源代码如下:

\begin{lstlisting}
  \begin{figure}[htbp]
    \centering
    \includegraphics[width=\textwidth{},keepaspectratio]{sun.jpg}
    \bicaption[fig:sun]{ 图 }{ 最左侧是太阳,向右依序为水星、金
      星 }{Fig.}{Outward from the Sun, the planets are Mercury, Venus,
      Earth, Mars, Jupiter, Saturn, Uranus and Neptune.}
  \end{figure}
\end{lstlisting}

可以看到,图~\ref{fig:sun}的宽度指定为版芯的宽度,然后使用了保持宽高比这
个选项。


\subsection{双图并列}

温文敦厚的新郎,美丽可爱的新娘。

Alan和Cher这一对从校园时代就相偎相依一直到步入婚礼的殿堂。

我一直觉得,这样的情侣是最最难得的,两个人之间最珍贵的东西得以一直保存、
延续,直至在未来某个时候升华成为生命中不可名状的一种记忆和体验。这个世界
上有太多因为坚持或者不坚持,执着或者不执着导致的有始无终。能够沿路陪伴,
最终成为眷属,也算是不大不小的奇迹。

两个人的婚礼誓词很肉麻,很感人,写在小纸片上,认真的读出来,直到读到对方
流下感动的泪水,直到在场的嘉宾都用掌声回应这份真情\footnote{蚌病生珠:罗
  兰湖畔}。

\begin{figure}[htbp]
  \centering
  \begin{minipage}{0.4\textwidth}
    \centering
    \includegraphics[keepaspectratio]{lang.jpg}
    \bicaption[fig:lang]{图}{新郎}{Fig.}{Bridegroom}
  \end{minipage}
  \begin{minipage}{0.4\textwidth}
    \centering
    \includegraphics[keepaspectratio]{liang.jpg}
    \bicaption[fig:niang]{图}{新娘}{Fig.}{Brige}
  \end{minipage}
\end{figure}

\begin{lstlisting}
  \begin{figure}[htbp]
    \centering
    \begin{minipage}{0.4\textwidth}
      \centering
      \includegraphics[keepaspectratio]{lang.jpg}
      \bicaption[fig:lang]{ 图 }{ 新郎 }{Fig.}{Bridegroom}
    \end{minipage}
    \begin{minipage}{0.4\textwidth}
      \centering
      \includegraphics[keepaspectratio]{liang.jpg}
      \bicaption[fig:niang]{ 图}{ 新娘 }{Fig.}{Brige}
    \end{minipage}
  \end{figure}
\end{lstlisting}


如果想要两幅并排的插图各有自己的标题,可以在 figure 环境中使用两
个 \texttt{minipage} 环境,每个里面插入一幅图 (见图~\ref{fig:lang}和
图~\ref{fig:niang}) 。不用 \texttt{minipage} 的话,因为插图标题的缺省宽度是
整个行宽;两幅插图就会上下排列。

这里指定了每个\texttt{minipage}的宽度为0.4倍的版芯宽度。当然,也可以自
己指定,只是两个宽度加起来不超过版芯宽度就可以了。


\subsection{两子图并列}

有你在,开水瓶里永远都有水喝;冰箱里永远有一袋应急的速冻饺子;阳台的晾衣
架上我昨天换下的衣服已经有了今天太阳的味道;小猫也不用害怕得病和不舒服,
它们有最负责任的家庭医生;还有别人根本见都没见过的那些点心和饼干;还有,
你是我见过的少有的照片和本人都好看的女孩\footnote{蚌病生珠:Two-year
  anniversary}。


\begin{figure}[htbp]
  \centering
  \subfigure[超人A]{
    \label{fig:1a}
    \includegraphics[keepaspectratio]{chao.jpg}
  }
  \hspace{20pt}
  \subfigure[超人A]{
    \label{fig:1b}
    \includegraphics[keepaspectratio]{ren.jpg}
  }
  \bicaption[fig:judy]{图}{小超人老师}{Fig.}{Judy}
\end{figure}

\begin{lstlisting}
  \begin{figure}[htbp]
    \centering
    \subfigure[超人A]{
      \label{fig:1a}
      \includegraphics[keepaspectratio]{chao.jpg}
    }
    \hspace{20pt}
    \subfigure[超人A]{
      \label{fig:1b}
      \includegraphics[keepaspectratio]{ren.jpg}
    }
    \bicaption[fig:judy]{图}{小超人老师}{Fig.}{Judy}
  \end{figure}
\end{lstlisting}

如果想要两幅并排的图片共享一个标题,并且各有自己的子标题,可以使
用\texttt{subcaption}宏包。如图~\ref{fig:judy},子图的标题用命令
\texttt{subcaption}即可。
%
%
%
%\section{参考文献}
%
%
%硕士论文写了3周。90多页英文,昏天黑地没日没夜写到想吐。好在有几个欧洲博士
%后帮忙改语法错。改的他们也很想哭。后来已经功成名就论文无数ACM
%Fellow英国Fellow of Royal Society的老板来给我们讲,写论文最重要的是
%写Introduction。写Introduction就和写童话一样\footnote{珵cici:硕士论文你有
%  哪些经验与收获?耗時多久?}。
%
%\begin{enumerate}[1.]
%\item 有一条巨龙抓走了公主 (介绍你的问题为什么值得研究)
%\item 巨龙是多么多么多么难打(强调你的研究的重要性)
%\item 王子提着一把金光闪闪的剑而不是破斧子烂长矛登场(你的方法好在哪里,别人sui在哪里)
%\item 王子是如何打败巨龙(你的方法简介)
%\item 从此王子和公主幸福的生活在一起。(解决了问题)
%\end{enumerate}
%
%老板说写论文就是写童话嘛。其余的也不过就是把这些东西细节讲一讲。做研究很
%简单的。听完我就不想再做研究了。合着我写到吐血掉头发的时候大牛都把写论文
%当给小盆友写童话。
%
%
%我们的一切知识都是从经验开始,这是没有任何怀疑的;
%因为,如果不是对象激动我们的感官,一则由它们自己引起表象,一则使我们的知性活动运作起来,对这些表象加
%以比较,把它们粘结或分开,这样把感性印象的原始素材加工成称之为经验的对象
%知识,那么知识能力又该由什么来唤起活动呢?所以
%按照时间,我们没有任何知识是先行于经验的,一切知识都是从经验开始的。
%
%只要是中文文献,图书,期刊,会议,专利等等需要为每个条目增加一个域:
%\begin{lstlisting}
%  language={c},
%\end{lstlisting}
%
%对于参考文献[1],原先的bib文件是这样的:
%\begin{lstlisting}
%  @article{ 李秋零1999 ,
%    title={ 康德何以步安瑟尔谟的后尘? },
%    author={ 李秋零 },
%    journal={ 中国人民大学学报 },
%    volume={2},
%    year={1999}
%  }
%\end{lstlisting}
%
%
%但是由于是中文文献,需要增加一个语言域,就变成下列样式:
%\begin{lstlisting}
%  @article{ 李秋零1999,
%    title={ 康德何以步安瑟尔谟的后尘? },
%    author={ 李秋零 },
%    language={c},
%    journal={ 中国人民大学学报 },
%    volume={2},
%    year={1999}
%  }
%\end{lstlisting}
