% !Mode:: "TeX:UTF-8"
% !TEX TS-program = XeLaTeX
% !TEX encoding = UTF-8 Unicode
\rhead{\kai\wuhao \kai\wuhao\leftmark}
\chapter{模版使用说明}
\label{chap01}

\section{个人信息}
使用模版的第一步当然是修改您的个人信息。与个人信息有关的内容位
于~\menu[,]{...,preface,cover.tex}~文件中。直接对照模板相应位置的内容,对其中的各个项目进行修改,填
写专业、姓名和导师的时候注意添加适当空格,也就$\sim$字符,以保持段落对齐。默认完成时间是最后一次编译main.tex的日期,但是也可以自定义。

\section{中英文摘要、关键字}
中英文摘要和关键字也位于~\menu[,]{{...},Preface,Cover.tex}~文件中,分别定义
在cabstract, eabstract, ckeywords, ekeywords等变量中,替换成与自己有关的,按前述方法编译得到新的论文PDF文件。

附:研究生院对摘要和关键字的要求:\kai{
\begin{asparaenum}
\item “摘要”是摘要部分的标题,不可省略。论文摘要是学位论文的缩影,文字
  要简练、明确。内容要包括目的、方法、结果和结论。单位制一律换算成国际标
  准计量单位制,除特殊情况外,数字一律用阿拉伯数码。文中不允许出现插图,
  重要的表格可以写入;
\item 关键词请尽量用《汉语主题词表》等词表提供的规范词。关键词之间用全角
  分号间隔,末尾不加标点;
\item 英文摘要和中文摘要对应,但不要逐字翻译。英文关键字使用半角分号间隔,
  末尾同样不加标点。
\end{asparaenum}
}\song

\section{正文}
正文部分包括引言:\menu[,]{...,body,chap00.tex}、正文内容章节:\menu[,]{...,body,chap01.tex}、\menu[,]{...,body,chap02.tex}、...和结论:\menu[,]{...,body,conclusion.tex}三部分,如果结论前要按序号顺序继续增加几章,可在\menu[,]{...,main.tex}文件中,把目前被注释掉的\menu{chap04.tex}$\sim$\menu{chap06.tex}取消注释,需要几章就取消注释几章。
如果还需要继续增加整章,可查看\menu{main.tex}中关于章内容增加书写格式,按如下代码继续增加:
\begin{lstlisting}
	\include{body/chap0n}
\end{lstlisting}
注意要同时把新增的章文件放在body文件夹内,如:\directory{.../body/chap07.tex},该文件句首复制如下代码之后再开始写正文:
\begin{lstlisting}
	% !Mode:: "TeX:UTF-8"
	% !TEX TS-program = XeLaTeX
	% !TEX encoding = UTF-8 Unicode	
	\chapter{第chap07这章的标题}
	\label{chap07}
\end{lstlisting}

所有图片放在\menu[,]{...,figures}文件夹中,调用方式见第\ref{chap02:figure}节,换句话说:图片fig.jpg放在前述文件夹,然后按第\ref{chap02:figure}节代码在相应位置插入正文,以索引命令对图片插入代码的bicaption标题label进行索引,即得到所需“如图...所示”的表述。当然,编号是自动变动的,按前述编译,全文的图、表、公式等都可自动协调变化——\TeX{}写作的优势之一。Word能以插入题注方式实现图、表引用的类似功能,但套格式和手动排版需要一定工作量,此外,公式需要借助第三方公式编辑器mathtype实现参考编号的双击引用,相对而言,\TeX{}这方面具备公认的优势。

\section{参考文献}
关于参考文献的增加、推送、自动编号引用,是本章内容,也是整个模板中最主要的部分,据我们所知,目前其他高校的论文模板中尚无采用Bib\LaTeX{}进行文献编译的,因此过程讲得详细一些。
\subsection{文献管理软件Jabref}
\subsubsection{软件下载与安装}
首先,最基础的内容是拥有自己的文献,然后才能谈得上对文献条目的管理。这部分并非新创,目前的文献管理软件已经比较多,例如火狐浏览器下的插件zotero、云存储功能强大的mendeley、对中文文献条目管理较好的notexpress,更不用提大名鼎鼎的EndNote。本次模板内容中,则介绍一款文献管理的自由软件:Jabref。

软件来龙去脉就不再啰嗦,感兴趣请移步互联网自行搜索,在下载和安装Jabref之前,需要先安装Java环境(\href{https://www.java.com/zh_CN/download/chrome.jsp}{DownLoad Page}),这是因为Jabref是用Java编写的,因此对Java基本库文件和环境有依赖性,下载和安装最新的Java版本之后,再下载和安装Jabref(\href{https://www.fosshub.com/JabRef.html}{DownLoad Page})的最新版本,二者都是免费软件,安装方面无需多说。
\subsubsection{基本的选项设置}
安装后双击桌面Jabref图标,在打开的初始界面中,依次单击菜单栏\menu[,]{options,preferences,general},将language改为“simplified Chinese”,下方编码格式改为“GB2312”;再在左侧栏目中找到“Appearance”,也就是“外观”栏,单击打开,右侧找到“设置表格字体”,新打开的界面中,设为“细明体”,或其他支持中英文的字体(如果不这么做,文献栏目后续新增加的中文文献将全部是空心方块乱码)。

如果使用的编译前端是\TeX{}Studio,则在Jabref的菜单栏依次单击:\menu[,]{选项,首选项,外部程序},单击\TeX{}Studio按钮,点击“浏览”按钮指定\TeX{}Studio的安装路径,退出设置。
\subsubsection{添加文献}
首先,本节内容的有效性只在教育网内用户得以确保,毕竟外网无包库属性,如果是在东大校园网内,应用本节方法添加文献肯定是有效可行的。此外,每个人在学术生涯中,通常只需要一个文献库,这是个人自定义的学术研究工作电子图书阅览室,后续工作主要是不断累积阅读文献,并将有价值的文献导入这个文献库,随着学术生涯的时间累积沉淀,该文献库经有序维护,将不断增加条目和大类,也让后续的文章、论文写作变得越来越轻松和正规。

对中文文献,以知网为例,按照如下顺序进行文献添加:
\begin{asparaenum}
	\item 在主页用关键词搜索出文献,搜索页面选择所需的文献条目,即在文献最左侧复选框内打钩,单击上方“导出/参考文献”按钮进入所选文献集中页面;
	\item 二次选中所需文献,再单击该页面上方的“导出/参考文献”按钮进入文献索引条目页面,选择其中的EndNote格式,单击复制到剪贴板;
	\item 随便找个位置(例如桌面)新建GB2312编码形式的txt文本,例如:\menu[,]{...,ChsRefItem.txt},把前一步剪贴板内的文献格式文本粘贴进该文件;
	\item 下载文献格式转换工具cvtCNKI.exe(\href{https://code.google.com/archive/p/cvtcnki/downloads}{DownLoad Page}),下载后双击exe可执行文件,“文献类型”中选择EndNote,“输出格式”中选择“Bib\TeX ”,点击界面上的“文件”按钮,选择前一步创建的格式文本\menu[,]{...,ChsRefItem.txt},同路径下将自动生成一个同名Bib\TeX 文件\menu[,]{...,ChsRefItem.bib},右键以记事本打开,复制全部条目,依次\menu[,]{右键主文献库bib文件,末尾粘贴,保存退出}。
\end{asparaenum}

对英文文献,各个网站一般都有Bib\TeX 的文献格式下载,按上述方法复制到自己的主文献库即可。如果想要让Jabref文献库的文献条目与文件正文链接,则可在硬盘上某个区域内,通常会把主文献库和附件放在同一个特定的文件夹,例如路径:\directory{.../References/AttachedFiles}内,在文献条目上\menu[,]{右键,附加文件,...},其余就不用多说了。

\subsubsection{文献的推送与引用}
为示例起见,模板内放了一个论文用的文献库\menu[,]{...,Ref.bib},所谓文献推送,指的是如何把刚刚建立的Jabref主文献库中文献,放入论文内进行自动的编号和引用,并在文内指定位置(通常是正文后的参考文献部分)将文献条目按照国家规定或者中外文期刊所规定的著录形式罗列。

首先谈谈文献的推送,所谓文献推送指的是在Jabref中选择需要插入论文内的文献条目,可以是单条,也可以是多条,选中后单击Jabref软件工具栏右侧的“推送到\TeX{}Studio”,则\TeX{}Studio中,光标焦点处就增加相应的文献引用条目。按biber文献编译的解释机制,\hologo{XeLaTeX}+Biber+\hologo{XeLaTeX},得到所需论文文档。

同样,附上研究生院对正文的要求:

\kai{
	“正文”不可省略。
	
	正文是硕士学位论文的主体,要着重反映研究生自己的工作,要突出新的见解,例
	如新思想、新观点、新规律、新研究方法、新结果等。正文一般可包括:理论分析;
	试验装置和测试方法;对试验结果的分析讨论及理论计算结果的比较等。
	
	正文要求论点正确,推理严谨,数据可靠,文字精练,条理分明,文字图表清晰整
	齐,计算单位采用国务院颁布的《统一公制计量单位中文名称方案》中规定和名称。
	各类单位、符号必须在论文中统一使用,外文字母必须注意大小写,正斜体。简化
	字采用正式公布过的,不能自造和误写。利用别人研究成果必须附加说明。引用前
	人材料必须引证原著文字。在论文的行文上,要注意语句通顺,达到科技论文所必
	须具备的“正确、准确、明确”的要求。
}\song
